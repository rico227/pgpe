\section{Prosa}\label{prosa}
Hier text juhu

\section{Einleitung}
- ABHOLUNG\\
- Was ist agile SW-Entw.\\
- Wieso wird agile Softwareentwicklung heute eingesetzt? + evtl. kurzer Vergleich zu V-Modell\\ 
- Was erhofft man sich vom Einsatz agiler Entwicklungsmethoden \\
- Was zeichnet agile Softwareentwicklung aus (allgemein)? - Was wird bezweckt?  \\
\section{Einordnung in den Produktentstehungsprozess}
- Eigene Prozessabläufe wurde gebildet, die sich nur mit Softwareentwicklung befassen\\
- Einordnen in Bild 2.9 + evtl. neuer Balken\\
- Embedded-Softwareentwicklung teilweise mit "Entwicklung Konstruktion E/E"-Balken verbunden\\
- Restliche Softwareentwicklung eher eigner Balken\\
\section{Beispiel Prozessablauf Agile}
- Beispiel des SAFe Scaled Agile Framework\\ 
- Wie wird agile Softwareentwicklung im Unternehmen umgesetzt? \\
- z.b. Wie laufen die Prozesse SAFe ab, Wann ist wer beteiligt (Buisness owner -> product owner -> entwickler\\
- Methoden wie Scrum, Ticketsystemen (Umsetzung), DevOps etc. \\ 
- Hier wird beschrieben wie sich Umternehmen bzw. Unternehmensbereich strukturiert (Abläufe/Prozesse/Schnittstellen/Personengruppen und ihre Aufgaben/Schnittstellen) \\ 
 
\section{Vergleich zu herkömmlichen Prozessen}

\subsection{Agile gegenüber Lasten/Pflichtenheft}
- Vergleich Agile / Lasten/Pflichtenheft Skript Kapitel 3.4 -> zum Beispiel auf Terminplanung und Komponentenbeschreibung eingehen\\

\subsection{Besonderheiten der SW-Entwicklung für automobile Anwendungen}
- ASPICE, höhere Sicherheitsanforderungen, Änderungen einpflegen, Testing Prozess (automatisiert, verschiedenen Ebenen, Von Unittest - bis Fahrzeugerprobung, sicherheitsrelevant\\

\section{Fazit}
- RÜCKFÜHRUNG\\
- Zusammenfassung, Positionieren zur agilen Entwicklungsmethoden \\
- Ausblick: wird weiter entwicklet, recht neu, viele Unternehmen übernehmen zur Zeit agile Methoden/Strukture \\
- Erfordert Umstellung von Denkweisen/Mitarbeitern/Organisationen\\
\section{Ausblick}