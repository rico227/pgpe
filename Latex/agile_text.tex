\section*{Prosa}\label{prosa}
Agile Softwareentwicklung ist ein Softwareentwicklungsprozess, der Flexibilität und Zusammenarbeit betont. Es ist iterativ und inkrementell, wobei der Schwerpunkt auf der Bereitstellung funktionierender Software am Ende jedes Zyklus liegt. Im Gegensatz zu einem Wasserfall-Prozess kann deutlich schneller funktionierende Software entwickelt werden.\\

Die Automobilindustrie ist eine der wettbewerbsintensivsten Branchen der Welt, in der zahlreiche Akteure agieren. Die Fähigkeit in der Branche, schnell auf Veränderungen zu reagieren, sowohl intern als auch extern, ist entscheidend für langfristigen Erfolg. In modernen Autos werden die meisten Funktionen von Software kontrolliert. Dies umfasst nahezu alle Bereiche vom Antriebsstrang über Infotainmentsysteme bis zum automatisierten Fahren. Agile Softwareentwicklung kann Unternehmen dabei helfen, diesen Bedarf zu decken, indem sie Agilität in allen Phasen ihres Produktlebenszyklus bietet. Die Einführung von Agile wurde durch die Notwendigkeit vorangetrieben, Entwicklungszeit und -kosten zu reduzieren sowie die Qualität und Reaktionsfähigkeit auf sich ändernde Kundenanforderungen zu verbessern. \cite{Schlosser2016} \cite{katumba2014}\\

Der Vortrag \glqq\titleDocument\grqq{} soll die agile Softwareentwicklung in den Produktentstehungsprozess einordnen und den Vergleich zu herkömmlichen Prozessen im automotive Bereich zeigen. An einem Beispiel soll ein agile Prozessablauf gezeigt werden, bevor ein Fazit und Ausblick gewagt wird.\\

\section*{Agenda}
\begin{enumerate}
	\item Einleitung Agile Softwareentwicklung
	\item Einordnung in den Produktentstehungsprozess
	\item Agiler Prozessablauf an einem Beispiel
	\item Agil im Vergleich zu herkömmlichen Prozessen
	\item Fazit
	\item Ausblick
\end{enumerate}