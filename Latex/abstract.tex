\section*{Kurzfassung}

Das Ziel der vorliegenden Arbeit war es, ein neues Architektur-Konzept für \acp{HMI} zu entwickeln und dieses in einem Proof-of-Concept zu verifizieren. Das Konzept basiert auf einer Gewichtung der Qualitätskriterien aus der ISO 25010 durch verschiedene Experten. Mit Hilfe dieser Gewichtung konnten Entwurfsmuster ausgewählt werden, die praktikabel für die Architektur sind. Aus den verschiedenen Entwurfsmustern entstand anschließend eine Architektur nach dem \ac{MVC}-Muster. Das entstandene Konzept legt großen Wert auf Modularisierbarkeit und Wiederverwendung von einzelnen Modulen. Mit Hilfe von Use Cases wurde Qt als \ac{HMI} Framework ausgewählt, mit dem das Konzept umgesetzt wurde. Die Umsetzung zeigte, dass das Konzept realisierbar ist und große Teile des Architektur-Konzepts übernommen werden können. Zusätzlich ergab sich die Möglichkeit, dass sich Projekte durch diesen Aufbau auch zusammenklicken lassen. Abschließend lässt sich sagen, dass das Konzept, wie gewünscht, ein hohes Maß an Modularität aufweist. Dadurch ergibt sich ebenfalls die Möglichkeit, große Teile wiederzuverwenden.\\
  


% einfacher Zeilenabstand
\singlespacing

% Inhaltsverzeichnis anzeigen
\newpage